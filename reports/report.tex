\documentclass[a4paper, 12pt]{article}
\usepackage[utf8]{inputenc}
\usepackage[russian]{babel}
\usepackage{graphicx}
\usepackage{float}
\usepackage{hyperref}
\usepackage{geometry}
\usepackage{amsmath}
\usepackage{booktabs}

\geometry{left=2.5cm, right=1.5cm, top=2cm, bottom=2cm}

\title{Отчет по проекту: Проверка гипотез с использованием случайных графов}
\author{Алметов Кирилл и Хорошенко Дмитрий}
\date{\today}

\begin{document}

\maketitle
\tableofcontents

\section{Введение}
В данной работе исследуется применение характеристик случайных графов для проверки гипотез согласия между распределениями:
\begin{itemize}
    \item $H_0$: $\xi \sim \text{Laplace}(0, \sqrt{0.5})$
    \item $H_1$: $\xi \sim \text{SkewNormal}(1)$
    \item $H_0$: $ \Gamma (\frac{1}{2}, \sqrt{\frac{1}{2}})$
    \item $H_1$: Weibull$(\frac{1}{2}; \lambda); \lambda_0 = \frac{1}{\sqrt{10}}$
\end{itemize}

Анализируются два типа графов:

\begin{itemize}
    \item KNN-графы
    \item Дистанционные графы
\end{itemize}

Характеристики:

\begin{itemize}
    \item Число компонент связности
    \item Хроматическое число
    \item Минимальная степень вершин
    \item Кликовое число
\end{itemize}

\section{Описание кода}
\subsection{Используемые инструменты}
\begin{itemize}
    \item Язык программирования: Python 3.10
    \item Основные библиотеки: numpy, networkx, matplotlib, scikit-learn, scipy, pandas
    \item Система контроля версий: Git
    \item Среды разработки: Jupyter Notebook, PyCharm
\end{itemize}

\subsection{Реализованные алгоритмы}
\subsubsection{Построение графов}
\begin{itemize}
    \item \textbf{KNN-граф}: Для каждой точки находим $k$ ближайших соседей
    \item \textbf{Дистанционный граф}: Соединяем точки на расстоянии $\leq d$
\end{itemize}

\subsubsection{Вычисление характеристик}
\begin{itemize}
    \item \textbf{Число компонент связности}: Алгоритм поиска связных компонент
    \item \textbf{Хроматическое число}: Жадный алгоритм раскраски графа
\end{itemize}

\subsubsection{Критерий согласия}
\begin{itemize}
    \item \textbf{Построение множества A}: Определение критической области при $\alpha=0.05$
    \item \textbf{Оценка мощности}: Вероятность правильного отклонения $H_0$
\end{itemize}

\section{Описание экспериментов}
\subsection{Эксперимент 1: Зависимость характеристик от параметров распределений}
\subsubsection{Цель}
Исследовать поведение характеристик при изменении параметров распределений ($\beta$ для Laplace, $\alpha$ для SkewNormal).

\subsubsection{Результаты}
\begin{figure}[H]
    \centering
    \includegraphics[width=0.9\textwidth]{params_vs_distributions.png}
    \caption{Зависимость числа компонент и хроматического числа от параметров распределений}
\end{figure}

\begin{itemize}
    \item Для распределения Лапласа: число компонент в KNN-графе $\approx$1, в дистанционном растет с $\beta$
    \item Для SkewNormal: при $\alpha=0$ поведение аналогично нормальному распределению
    \item Хроматическое число в KNN-графах стабильнее, чем в дистанционных
\end{itemize}

\subsubsection{Результаты}
\begin{figure}[H]
    \centering
    \includegraphics[width=0.9\textwidth]{params_vs_distributions2.png}
    \includegraphics[width=0.9\textwidth]{params_vs_distributions3.png}
    \caption{Зависимость минимальной степени и кликового числа от параметров распределений}
\end{figure}

\begin{itemize}
    \item Минимальная степень остаётся постоянной
    \item Кликовое число уменьшается с ростом u
\end{itemize}

\subsection{Эксперимент 2: Зависимость от параметров графов и размера выборки}
\subsubsection{Цель}
Исследовать поведение характеристик при изменении параметров графов ($k$ для KNN, $d$ для дистанционных) и размера выборки $n$.

\subsubsection{Результаты}
\begin{figure}[H]
    \centering
    \includegraphics[width=0.9\textwidth]{params_vs_graphs.png}
    \caption{Зависимость характеристик от параметров графов и размера выборки}
\end{figure}

\begin{itemize}
    \item Число компонент уменьшается с ростом $k$ (KNN) и $d$ (дистанционные), что и ожидалось.
    \item Хроматическое число растет с увеличением $n$ для обоих типов графов
    \item В KNN графах число компонент растет с увеличением $n$
\end{itemize}

\subsubsection{Результаты}
\begin{figure}[H]
    \centering
    \includegraphics[width=0.9\textwidth]{params_vs_graphs2.png}
    \caption{Зависимость характеристик от параметров графов и размера выборки}
\end{figure}

\begin{itemize}
    \item Всё растёт линейно.
\end{itemize}


\subsection{Эксперимент 3: Построение критических областей}
\subsubsection{Цель}
Построить критические области для проверки гипотез при $\alpha=0.05$.

\subsubsection{Результаты}
\begin{figure}[H]
    \centering
    \includegraphics[width=0.9\textwidth]{critical_regions_fixed.png}
    \caption{Распределения характеристик с критическими областями}
\end{figure}


\subsection{Эксперимент 4: Классификация с несколькими характеристиками}
\subsubsection{Цель}
Построить классификатор, использующий обе характеристики для различения распределений.

\subsubsection{Результаты}
\begin{figure}[H]
    \centering
    \includegraphics[width=0.8\textwidth]{roc_curves.png}
    \caption{ROC-кривые для классификаторов (n=100)}
\end{figure}

\begin{table}[H]
\centering
\caption{Точность классификации (n=500)}
\begin{tabular}{lcc}
\toprule
Модель & Дистанционный граф \\
\midrule
Random Forest &  0.94 \\
SVM & 0.93 \\
\bottomrule
\end{tabular}
\end{table}

\subsection{Эксперимент 5: Анализ важности характеристик}
\subsubsection{Цель}
Исследовать важность характеристик как признаков классификации.

\subsubsection{Результаты}
\begin{figure}[H]
    \centering
    \includegraphics[width=0.9\textwidth]{feature_importance.png}
    \caption{Важность характеристик для классификации}
\end{figure}
\begin{figure}[H]
    \centering
    \includegraphics[width=0.9\textwidth]{metrics.png}
    \caption{Точность и ошибки}
\end{figure}

\begin{itemize}
    \item Число компонент связности - наиболее важный признак
    \item Для дистанционных графов важность признаков выше
    \item С ростом $n$ важность числа компонент уменьшается для дистанционных графов, при больших n он становится не так важен, в отличии от хроматического числа.
\end{itemize}

\section{Промежуточные выводы}
\begin{enumerate}
    \item Дистанционные графы показывают лучшие результаты (мощность до 0.725)
    \item Число компонент связности - наиболее информативная характеристика
    \item Random Forest превосходит SVM по всем метрикам
    \item С ростом размера выборки:
    \begin{itemize}
        \item Точность увеличивается до 94\% 
        \item Дисперсия метрик уменьшается
    \end{itemize}
\end{enumerate}

\section{Заключение}
Основные результаты исследования:
\begin{itemize}
    \item Разработан эффективный метод проверки гипотез с использованием характеристик случайных графов
    \item Для практического применения рекомендуются:
    \begin{itemize}
        \item Дистанционные графы с $d \approx 0.5$
        \item Число компонент связности как основная характеристика
        \item Random Forest в качестве классификатора
    \end{itemize}
    \item Наилучшие результаты достигнуты при $n \geq 100$ (мощность $> 0.9$)
\end{itemize}

Перспективные направления:
\begin{itemize}
    \item Добавление других характеристик графов
    \item Адаптивный выбор параметров графов
    \item Применение графовых нейронных сетей
\end{itemize}

\end{document}
